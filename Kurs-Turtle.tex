\documentclass[9pt]{article}
\usepackage{ngerman}
\usepackage[utf8]{inputenc}
\usepackage{amsmath}
\usepackage{amsthm}
\usepackage{amssymb}
\usepackage{amsfonts}
\usepackage{mathrsfs}
\usepackage{stmaryrd}
\usepackage{enumerate}
\usepackage{listings}
\usepackage{color}
\usepackage{float}
\usepackage{mathtools}
\usepackage{fontawesome}
\usepackage{csquotes}
\usepackage{gensymb}
\usepackage{tikz}
\usepackage[margin = 2cm]{geometry}
\usepackage{verbatim}
\usepackage{hyperref}
\usetikzlibrary{calc}



\author{Niklas Schneider - Maximilian Krahn}
\date{}

\newcommand{\setFW}{
	\lstset{ %this is the stype
    mathescape=true,
    %frame=tB,				
    %numbers=left,
    %numberstyle=\tiny,		
    basicstyle=\ttfamily,
    keywordstyle=\color{blue}\bf,
    resetmargins=true,
    language = python,
    %xleftmargin=.04\textwidth,
    numbersep=0pt,
    tabsize=4
}
}
\lstnewenvironment{code}
{
	\setFW
}
{}

\lstMakeShortInline[
mathescape=true,
%frame=tB,				
%numbers=left,
%numberstyle=\tiny,		
basicstyle=\ttfamily,
keywordstyle=\color{blue}\bf,
resetmargins=true,
language = python,
%xleftmargin=.04\textwidth,
numbersep=0pt,
tabsize=4]@


\title{CoderDojo Saar - Turtle}
\date{05.03. - 06.03.2021}


\begin{document}
	\maketitle
		
	\begin{center}
		{\large Ablaufplan}
	\end{center}	
	
	\tableofcontents
	
	
	\section{Erster Tag}
	\subsection{Einführung ($10$ min)} 
	\begin{itemize}
		\item
		Einführung, Begrüßung, was machen wir heute?
		
		\item 
		\textbf{Repo klonen} über GitHub $\to$ Projekt über GitHub Link importieren.
		
		\item
		\textbf{Einführung in Repl}:
		\begin{itemize}
			\item 
			Links Dateien und Aufgaben, Mitte oben Run Knopf, rechts Ausgabe.
			
			\item
			Wenn abstürzt, Tab neu laden, speichert automatisch.
		\end{itemize}	
	\end{itemize}


	\subsection{Aufgabe 1 ($5+20=25$ min)}
	\begin{enumerate}[(a)]
		\item 
		\begin{itemize}
			\item 
			Einführung Befehle: @left, right, forward, back@
			\item
			Zeichnen einfaches Quadrat zusammen.
			\item
			Zeichnen daneben ein Quadrat in die andere Richtung.
			\item
			Dann rückwärts den Pfad abgehen ohne eine neue Linie zu zeichnen.
		\end{itemize}
		
		\item
		\begin{itemize}
			\item 
			Zeichne das Haus des Nikolaus mit den gerade gelernten Befehlen.
		\end{itemize}
	\end{enumerate}


	\subsection{Aufgabe 2 ($5+25=30$ min)}
	\begin{enumerate}[(a)]
		\item 
		\begin{itemize}
			\item 
			Einführung Befehle: @penup, pendown, color@
			\item
			Gehen zusammen die Buchstaben ''MK`` durch.
		\end{itemize}
		
		\item
		\begin{itemize}
			\item 
			Zeichne die eigenen Initialen bunt.
		\end{itemize}
	\end{enumerate}
		
	
	\subsection{Aufgabe 3 ($5+30 + 10=45$ min)}

	\section{Zweiter Tag}
	
	\subsection{Einführung Rekursion ($15$ min)}
	\begin{itemize}
		\item
		Neues Programmierprinzip: Rekursion.
		\item
		Beispiel: Aufsummieren der ersten $n$ Zahlen.
		\item
		Zuerst an @for@-Schleife zeigen.
		\item
		Dann dasselbe mit Rekursion.
	\end{itemize}

	\subsection{Aufgabe 4 ($10$ min)} 
	\begin{itemize}
		\item
		Schreibe eine rekursive Funktion, die $a^b$ berechnet und teste sie anhand verschiedener Eingaben.
		\item
		Besprechen.
	\end{itemize}

	\subsection{Aufgabe 5 ($10 + 20 + 10 = 40$ min)}
	\begin{itemize}
		\item 
		Kochkurve / Schneeflocke.
		\item
		\begin{enumerate}[(a)]
			\item 
			Schreibe eine Funktion, die eine einfache Kochkurve mit einer gegebenen Länge $l$ zeichnet.
			\item
			Schreibe eine Funktion, die das folgende Verhalten zeigt:
			\begin{itemize}
				\item 
				Sie nimmt zwei Parameter entgegen: Die Länge $l$ und die sogenannte Rekursionstiefe $n$, das heißt, die Anzahl der Schritte.
				\item
				Wenn nur noch ein Schritt übrig ist, soll die Funktion eine einfache Kochkurve der Länge $l$ zeichnen.
				\item
				Ansonsten soll sie an den Stellen, an denen es geradeaus geht, stattdessen kleinere Kochkurven zeichnen. Die Anzahl an Schritten wird dabei um eins verkleinert.				
			\end{itemize}
			\item
			Eine Schneeflocke besteht aus mehreren aneinandergereihten Kochkurven in der Form eines $n$-Ecks. Finde die Anzahl an Ecken heraus, für die die Schneeflocke am schönsten aussieht.
		\end{enumerate}
	\end{itemize}
	
	\subsection{Aufgabe 6 ($5 + 5 + 15 + 15 + 5 + 10$ min)}
	\begin{itemize}
		\item 
		Fibonacci-Kurve.
		\item
		Gemeinsam Definition von Fibonacci-Zahlen durchgehen und die ersten paar anzeigen lassen.
		\item
		\begin{enumerate}[(a)]
			\item 
			Schreibe eine Funktion, die ein Quadrat mit der Seitenlänge $l$ zeichnet.
			\item
			Schreibe eine Funktion, die das folgende Verhalten zeigt:
			\begin{itemize}
				\item 
				Sie nimmt zwei Parameter entgegen: Die Anzahl $n$ und ein Länge $l$
				\item
				Solange noch Schritte übrig sind, soll die Funktion im $i$-ten Schritt ein Quadrat mit der Seitenlänge $l\cdot\mathsf{fib}(i)$ zeichnen.
				\item
				Die Quadrate sollen spiralförmig gegen den Uhrzeigersinn um den Startpunkt angeordnet sein.
			\end{itemize}
			\item
			Zeichne die sogenannte Fibonacci-Kurve. Gehe dazu wie folgt vor:
			\begin{itemize}
				\item 
				Ähnlich wie oben nimmt die Funktion @kurve@ die Parameter @n@ und @l@ an.
				\item
				Ein Kurvensegment besteht immer aus einem Viertelkreis. Ein solcher lässt sich mit @circle(r, 90)@ zeichnen, wobei @r@ der Radius des Kreises ist.
				\item
				Im Fall unserer Kurve ist der Radius des $i$-ten Kurvensegments genau die $i$-te Fibonacci-Zahl.
			\end{itemize}
			\item
			Färbe die Quadrate bunt ein. Benutze dazu die Funktion @naechste_farbe@, die als Parameter den aktuellen Schritt annimmt. 
			
			Außerdem kannst du die Quadrate in der Funktion @quadrat(...)@ farbig ausfüllen. Benutze dazu die Turtle-Funktionen @begin_fill()@ und @end_fill()@.
			
			Denke daran, auch danach noch die Kurve einzufärben, damit man sie auf dem bunten Hintergrund noch erkennt.
		\end{enumerate}
	\end{itemize}

	\subsection{Abschluss / Puffer ($10$ min)}
	\begin{itemize}
		\item 
		Gegebenenfalls Kinder Bilder zeigen lassen.
		\item
		Verabschieden.
	\end{itemize}

	
	
\end{document}